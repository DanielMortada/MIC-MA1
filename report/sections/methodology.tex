% =====================================
% Methodology Section
% File: sections/methodology.tex
% =====================================

\section{Methodology}
This section describes our three-phase approach to musical instrument classification, including transfer learning with ResNet-18, the development of custom CNN architectures, and our flexible experimental framework.

\subsection{Phase 1: Transfer Learning with ResNet-18 vs. Custom CNN}
In the first phase of our project, we established a strong baseline using transfer learning and developed an initial custom architecture.

\subsubsection{Transfer Learning with ResNet-18}
As a baseline approach, we adopted the ResNet-18 architecture \cite{he2016deep}, which has demonstrated strong performance across various computer vision tasks. ResNet-18 features a relatively shallow architecture compared to other ResNet variants but includes skip connections (residual blocks) that facilitate gradient flow during training, making it particularly suitable for transfer learning.

Our transfer learning approach involved:
\begin{itemize}
    \item Initializing the network with weights pre-trained on ImageNet
    \item Replacing the final fully-connected layer with a new layer containing 30 output units (one per instrument class)
    \item Fine-tuning the entire network with a lower learning rate to preserve pre-learned features while adapting to the new task
\end{itemize}

The choice of ResNet-18 as a baseline was motivated by its efficient architecture, proven capability on image classification tasks, and reasonable computational requirements. The residual connections in ResNet-18 help mitigate the vanishing gradient problem, allowing for effective fine-tuning across all layers.

% TODO: Insert diagram showing ResNet-18 architecture
% Figure 2: Architecture of ResNet-18 showing the residual connections that enable
% efficient gradient flow during training.

\subsubsection{Custom CNN Architecture}
To explore alternatives to transfer learning, we designed a custom CNN architecture specifically for musical instrument classification. The design follows a pyramidal structure with progressively increasing channel depths and reducing spatial dimensions, similar to established CNN architectures but tailored to our specific task.

The custom CNN comprises five convolutional blocks with the following structure:
\begin{itemize}
    \item \textbf{Block 1:} 3 → 32 → 32 channels (basic edge and texture detection)
    \item \textbf{Block 2:} 32 → 64 → 64 channels (more complex patterns)
    \item \textbf{Block 3:} 64 → 128 → 128 channels (instrument parts and shapes)
    \item \textbf{Block 4:} 128 → 256 → 256 channels (high-level instrument features)
    \item \textbf{Block 5:} 256 → 512 → 512 channels (complex features)
\end{itemize}

Each block consists of two convolutional layers with batch normalization and ReLU activation, followed by max pooling for spatial dimension reduction and dropout for regularization. The progressive increase in channel depth allows the network to learn a hierarchy of features, from simple edges and textures in early layers to complex instrument-specific patterns in deeper layers.

Following the convolutional blocks, global average pooling reduces the feature maps to a 512-dimensional vector, which is fed into a classifier head consisting of a hidden layer, dropout for regularization, and a final output layer with 30 units corresponding to the instrument classes.

Key design principles incorporated in our custom architecture include:
\begin{itemize}
    \item \textbf{Progressive Feature Extraction:} Increasing channel depth through the network allows learning a hierarchy of features from simple to complex
    \item \textbf{Balanced Network Depth:} The architecture is deep enough to learn complex features but not so deep as to suffer from optimization difficulties
    \item \textbf{Regularization Techniques:} Batch normalization and dropout help prevent overfitting
    \item \textbf{Parameter Efficiency:} Small kernel sizes (3×3) and appropriate channel progression keep the model computationally efficient while maintaining expressivity
    \item \textbf{Global Average Pooling:} Reduces parameters compared to fully-connected layers and improves translation invariance
\end{itemize}

% TODO: Insert diagram illustrating the custom CNN architecture
% Figure 3: Visual representation of our custom CNN architecture showing the five
% convolutional blocks and classifier head.

\subsection{Phase 2: Flexible Experimental Framework and Architecture Comparison (In Progress)}
The second phase of our project extends beyond the comparison of a single custom model with ResNet-18 to explore multiple model architectures systematically. For this purpose, we developed a flexible experimental framework that enables controlled comparison of different CNN variants.

\subsubsection{Flexible Experimental Framework}
Our flexible experimental framework is designed to facilitate systematic comparison of different architectures while controlling for confounding factors. The framework provides:
\begin{itemize}
    \item Consistent data loading and preprocessing across all experiments
    \item Standardized training procedures with configurable hyperparameters
    \item Uniform evaluation metrics and visualization tools
    \item Automated logging of results for comparative analysis
    \item Modular architecture definition through configuration files
\end{itemize}

This approach enables us to isolate the effects of architectural changes from other variables, ensuring that performance differences can be attributed to design choices rather than implementation details. Our methodology is inspired by systematic architecture exploration approaches \cite{zoph2018learning, tan2019efficientnet} but tailored to our specific domain and constraints.

\subsubsection{Architecture Variants Under Investigation}
In the ongoing Phase 2, we are investigating several architectural variants of our custom CNN, including:
\begin{itemize}
    \item \textbf{Base Custom CNN:} The original architecture from Phase 1
    \item \textbf{Deeper CNN:} Adding additional convolutional blocks to increase depth
    \item \textbf{Wider CNN:} Increasing the number of filters in each layer
    \item \textbf{Regularized CNN:} Incorporating additional regularization techniques
    \item \textbf{ResNet-18:} Maintained as a standard baseline for comparison
\end{itemize}

For each variant, we are examining training dynamics, convergence speed, final accuracy, and computational efficiency. This systematic comparison will help identify the most promising architectural patterns for musical instrument classification.

% TODO: Insert table or diagram showing the key differences between architecture variants
% Figure 4: Comparison of architectural variants being evaluated in Phase 2.

\subsection{Phase 3: Deeper CNN Model Optimization}
The final phase of our project focused on optimizing the Deeper CNN architecture, which was identified as the best-performing custom model from Phase 2. Based on the analysis of architecture performance, we developed a systematic optimization strategy with four main components:

\begin{itemize}
    \item \textbf{Enhanced Data Augmentation:} Implementation of stronger augmentation techniques including random erasing and random crops to improve model robustness.
    
    \item \textbf{Architecture Refinements:} Fine-tuning the dropout strategy with progressive dropout rates, adding residual connections to improve gradient flow, and exploring channel attention mechanisms to focus on important features.
    
    \item \textbf{Training Strategy Optimization:} Implementation of the AdamW optimizer with optimized weight decay, OneCycle learning rate scheduling with appropriate warmup, and gradient clipping for training stability.
    
    \item \textbf{Experimental Techniques:} Exploration of label smoothing, longer training with early stopping, and hyperparameter optimization.
\end{itemize}

The optimization strategy was implemented incrementally, with careful monitoring of validation performance to guide adjustments. Our approach was designed to systematically improve the Deeper CNN's performance while maintaining its architectural benefits.

\subsubsection{Enhanced Architecture Components}
For the architectural enhancements, we implemented:

\begin{itemize}
    \item \textbf{Residual Connections:} Skip connections between convolutional blocks to improve gradient flow in the deep network, similar to those in ResNet but adapted to our custom architecture.
    
    \item \textbf{Progressive Dropout:} A carefully tuned dropout pattern [0.1, 0.15, 0.2, 0.25, 0.3, 0.4] that gradually increases through the network, providing stronger regularization in deeper layers where overfitting is more likely.
    
    \item \textbf{Channel Attention:} Optional attention mechanisms that allow the network to dynamically emphasize important feature channels based on their relevance to the classification task.
\end{itemize}

\subsubsection{Advanced Training Configuration}
Our training configuration was optimized with:

\begin{itemize}
    \item \textbf{AdamW Optimizer:} Used with a weight decay of 0.0005 for better regularization properties compared to standard Adam.
    
    \item \textbf{OneCycle Learning Rate:} Implemented with a maximum learning rate of 0.003 and 30\% of training spent in the warmup phase, allowing faster convergence with stable training.
    
    \item \textbf{Gradient Clipping:} Applied with a maximum norm of 2.0 to prevent exploding gradients and stabilize training.
    
    \item \textbf{Extended Training:} Allowed up to 75 epochs with a patient early stopping mechanism to ensure convergence without overfitting.
\end{itemize}

The goal of Phase 3 was to push the performance boundaries of our custom CNN architecture for musical instrument classification, building upon the insights gained in Phase 2 while maintaining interpretability and efficiency.
