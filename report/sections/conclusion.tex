% Conclusion Section
\section{Conclusion}
This paper presented a comprehensive three-phase approach to musical instrument classification using convolutional neural networks. In Phase 1, we compared transfer learning with a pre-trained ResNet-18 model against a custom CNN architecture developed from scratch. Our findings demonstrated that while transfer learning offers superior performance (99.33\% test accuracy) and faster convergence, custom architectures provide valuable insights into the feature learning process and reasonable performance (80.67\% test accuracy) with fewer parameters.

In Phase 2, we systematically compared multiple architectural variants using a flexible experimental framework. This comparison revealed that the Deeper CNN architecture (86.67\% test accuracy) outperformed other custom variants including Base CNN (85.33\%), Regularized CNN (81.33\%), and Wider CNN (80.67\%). These results demonstrated that increased depth provides more benefit than increased width for musical instrument classification, while highlighting the importance of balancing regularization with model capacity.

Phase 3 focused on optimizing the Deeper CNN architecture through a methodical enhancement process. We implemented residual connections, refined the dropout strategy, optimized the training procedure with AdamW and OneCycle learning rate scheduling, and carefully balanced data augmentation. These optimizations yielded significant improvements, with the optimized model achieving 92.33\% test accuracy—a 5.66\% improvement over the original Deeper CNN and substantially closing the gap with the transfer learning approach.

Our work provides several important contributions to the understanding of deep learning approaches for specialized image classification tasks:

\begin{itemize}
    \item A systematic methodology for developing, comparing, and optimizing custom CNN architectures for domain-specific tasks
    
    \item Empirical evidence on the relative importance of architectural components like depth, width, and regularization for musical instrument classification
    
    \item Insights into effective optimization strategies that can significantly improve custom model performance without resorting to transfer learning
    
    \item Visualization techniques that provide interpretable insights into how models learn to distinguish between instrument classes
\end{itemize}

The study also highlights the critical importance of incremental optimization and careful balance between model complexity and training stability. Our experience showed that simultaneous implementation of multiple optimization techniques can destabilize model training, while a methodical step-by-step approach leads to substantial performance gains.

Future work could explore ensemble methods combining different architectural variants, knowledge distillation from the ResNet-18 model to more compact architectures, and deployment optimizations for real-time musical instrument classification in resource-constrained environments. Additionally, the systematic approach developed in this work can be extended to other domain-specific image classification tasks beyond musical instrument identification.
