% Conclusion Section
\section{Conclusion}
This paper presented a three-phase approach to musical instrument classification using convolutional neural networks. In Phase 1, we compared transfer learning with a pre-trained ResNet-18 model against a custom CNN architecture developed from scratch. Our findings demonstrate that while transfer learning offers superior performance (100\% test accuracy) and faster convergence, custom architectures provide valuable insights into the feature learning process and reasonable performance (80.67\% test accuracy) with fewer parameters.

The significant performance gap between the two approaches highlights the power of transfer learning for specialized image classification tasks, particularly when working with limited dataset sizes. At the same time, the respectable performance of our custom model demonstrates that domain-specific architectures can achieve competitive results when properly designed and trained.

Phase 2 of our project, which is currently ongoing, extends this investigation by systematically comparing multiple architectural variants using a flexible experimental framework. This work aims to identify key architectural components that contribute to high performance in custom CNN models for this domain.

Phase 3, planned as future work, will focus on optimizing the most promising architecture identified in Phase 2. This will involve advanced training techniques, hyperparameter optimization, and potentially ensemble methods to push the boundaries of custom model performance.

This work contributes to the broader understanding of deep learning approaches for specialized image classification tasks and offers practical insights into the tradeoffs between transfer learning and custom architecture development. Our phased approach demonstrates a systematic methodology for developing and refining deep learning models, which can be applied to other domain-specific classification tasks beyond musical instrument identification.
