% =====================================
% Results and Discussion Section
% File: sections/results.tex
% =====================================

\section{Results and Discussion}

\subsection{Phase 1 Results: ResNet-18 vs. Initial Custom CNN}

\subsubsection{Baseline ResNet-18 Performance}
The ResNet-18 model with transfer learning achieved exceptional performance on the musical instrument classification task, reaching 100\% accuracy on the test set. This perfect accuracy was achieved after only 8 epochs of fine-tuning, demonstrating the effectiveness of transfer learning when applied to this domain. The total training time was approximately 11 minutes and 20 seconds.

The rapid convergence of ResNet-18 can be attributed to its pre-trained weights, which already encode general visual features learned from ImageNet. These features proved highly transferable to musical instrument classification, requiring only minor adjustments during fine-tuning to achieve optimal performance.

\subsubsection{Initial Custom CNN Performance}
Our custom CNN architecture, trained from scratch without any pre-trained weights, achieved a test accuracy of 80.67\%. This result, while lower than the ResNet-18 baseline, is still impressive considering the model was learning all features from the beginning without any prior knowledge.

Training the custom CNN took approximately 29 minutes, reaching its peak validation accuracy after 47 epochs. The learning curve showed steady improvement throughout training:
\begin{itemize}
    \item Early training (Epochs 1-10): Rapid improvement from 5.33\% to 34.67\% validation accuracy
    \item Mid training (Epochs 11-30): Continued learning with validation accuracy reaching 66.00\%
    \item Late training (Epochs 31-47): Fine-tuning of features with validation accuracy peaking at 83.33\%
\end{itemize}

The difference between validation (83.33\%) and test (80.67\%) accuracy indicates reasonably good generalization without significant overfitting.

% TODO: Insert learning curve figure comparing ResNet-18 and Custom CNN
% Figure 5: Learning curves showing validation accuracy versus training epoch for
% ResNet-18 and Custom CNN models during Phase 1.

\subsubsection{Comparative Analysis of Phase 1 Models}
Table I presents a comparison between the ResNet-18 and custom CNN models across various metrics.

% TODO: Insert Table I showing the model comparison metrics
% Table I: Comparison of ResNet-18 and Custom CNN performance metrics.
% | Model | Parameters | Test Accuracy | Training Time | Best Epoch | Input Size |
% |-------|------------|---------------|---------------|------------|------------|
% | ResNet-18 (Transfer Learning) | 11.7 million | 100% | ~11m 20s | 8 | 224x224 |
% | Custom CNN (From Scratch) | 8.6 million | 80.67% | ~29m | 47 | 224x224 |

Key differences observed between the two approaches include:
\begin{itemize}
    \item \textbf{Convergence Speed:} ResNet-18 converged much faster (best performance at epoch 8 vs. epoch 47), highlighting the advantage of transfer learning in reducing training time
    \item \textbf{Parameter Efficiency:} The custom CNN used fewer parameters (8.6M vs. 11.7M) while still achieving reasonable performance
    \item \textbf{Accuracy Gap:} The 19.33\% accuracy difference demonstrates the value of transfer learning for complex image classification tasks
    \item \textbf{Training Resources:} The custom CNN required more than twice the training time despite having fewer parameters
\end{itemize}

This comparison illustrates the tradeoff between leveraging pre-trained weights and building custom architectures. While transfer learning offers superior performance and faster convergence, the custom approach provides greater architectural control and insight into the feature learning process specific to musical instrument classification.

\subsection{Phase 2 Results: Architecture Comparison}
In Phase 2, we conducted a systematic comparison of multiple CNN architectures using our flexible experimental framework. This comparison allowed us to identify the most promising architecture for further optimization.

\subsubsection{Performance Comparison of Architecture Variants}
Table I presents the performance comparison of the different architecture variants evaluated in Phase 2.

\begin{table}[ht]
\caption{Performance Comparison of CNN Architecture Variants}
\centering
\begin{tabular}{lccccc}
\toprule
\textbf{Model} & \textbf{Test Acc.} & \textbf{F1 Score} & \textbf{Precision} & \textbf{Recall} & \textbf{Training Time} \\
\midrule
ResNet18 & 99.33\% & 0.9933 & 0.9944 & 0.9933 & 32.48 min \\
Deeper CNN & 86.67\% & 0.8567 & 0.8935 & 0.8667 & 36.06 min \\
Base CNN & 85.33\% & 0.8452 & 0.8685 & 0.8533 & 34.96 min \\
Regularized CNN & 81.33\% & 0.8034 & 0.8354 & 0.8133 & 38.39 min \\
Wider CNN & 80.67\% & 0.7973 & 0.8519 & 0.8067 & 62.29 min \\
\bottomrule
\end{tabular}
\end{table}

Key findings from the architecture comparison include:
\begin{itemize}
    \item \textbf{ResNet-18} significantly outperformed all custom architectures, demonstrating the power of transfer learning and pre-trained feature extractors.
    
    \item \textbf{Deeper CNN} emerged as the best-performing custom architecture with 86.67\% test accuracy, suggesting that increased depth provides better feature learning capability for instrument classification.
    
    \item \textbf{Base CNN} performed reasonably well at 85.33\% accuracy, serving as a strong baseline for our custom architectures.
    
    \item \textbf{Wider CNN} had the longest training time (62.29 minutes) but delivered the lowest performance among custom models, indicating that simply increasing network width is not an efficient approach for this task.
    
    \item \textbf{Regularized CNN} showed lower performance than the Base CNN, suggesting that excessive regularization might have limited the model's learning capacity.
\end{itemize}

\subsubsection{Architectural Insights}
The comparison revealed several important architectural insights:
\begin{itemize}
    \item \textbf{Depth vs. Width:} Increasing network depth (Deeper CNN: 86.67\%) provided more benefit than increasing width (Wider CNN: 80.67\%), with a significant 6\% performance gap between these approaches.
    
    \item \textbf{Regularization Effects:} While some regularization is necessary, the heavily regularized model (81.33\%) underperformed compared to models with standard regularization, highlighting the importance of balancing regularization with model capacity.
    
    \item \textbf{Precision-Recall Balance:} The Deeper CNN showed the best balance between precision (0.8935) and recall (0.8667), suggesting more robust generalizable learning.
    
    \item \textbf{Parameter Efficiency:} The Deeper CNN achieved higher accuracy despite longer training time, indicating that the additional complexity was effectively utilized for learning relevant features.
\end{itemize}

Based on these results, we selected the Deeper CNN architecture for further optimization in Phase 3.

\subsection{Phase 3 Results: Deeper CNN Optimization}
Following the architecture comparison in Phase 2, Phase 3 focused on optimizing the Deeper CNN model, which emerged as the best-performing custom architecture.

\subsubsection{Initial Optimization Challenges}
Our initial optimization attempts revealed important insights about the balance between model complexity and training stability. When we simultaneously implemented multiple optimization techniques—including residual connections, attention mechanisms, label smoothing, and strong regularization—the model became unstable during training, resulting in poor performance (only 6\% test accuracy).

This highlighted the importance of incremental optimization and careful monitoring of how each enhancement affects model behavior. Based on this experience, we redesigned our optimization strategy with a more methodical, step-by-step approach.

\subsubsection{Successful Optimization Strategy}
Our successful optimization approach included:

\begin{itemize}
    \item \textbf{Simplified Architecture Enhancements:} We began by implementing residual connections while disabling attention mechanisms, then carefully adjusted the progressive dropout pattern to [0.1, 0.15, 0.2, 0.25, 0.3, 0.4] to provide more gradual regularization.
    
    \item \textbf{Conservative Learning Dynamics:} We reduced the maximum learning rate from 0.01 to 0.003 in the OneCycle schedule, decreased weight decay from 0.001 to 0.0005, and made the learning rate schedule less aggressive with smaller division factors.
    
    \item \textbf{Improved Training Stability:} We increased early stopping patience from 15 to 25 epochs, reduced the early stopping delta for finer sensitivity to improvements, and increased the gradient clipping threshold from 1.0 to 2.0 to allow more gradient flow.
    
    \item \textbf{Moderated Data Augmentation:} We shifted from 'optimized' to 'medium' augmentation strength to prevent overly aggressive transformations that might hinder learning.
\end{itemize}

\subsubsection{Optimized Model Performance}
The optimized Deeper CNN model achieved a test accuracy of 92.33\%, representing a 5.66\% improvement over the original Deeper CNN (86.67\%). Key performance metrics include:

\begin{itemize}
    \item \textbf{Test Accuracy:} 92.33\% (compared to 86.67\% for the original Deeper CNN)
    \item \textbf{F1 Score:} 0.9233 (improved from 0.8567)
    \item \textbf{Precision:} 0.9254 (improved from 0.8935)
    \item \textbf{Recall:} 0.9233 (improved from 0.8667)
    \item \textbf{Best Validation Accuracy:} 94.00\% (at epoch 23)
    \item \textbf{Training Time:} 42.35 minutes
\end{itemize}

\begin{figure}[ht]
    \centering
    % TODO: Insert learning curve for optimized model
    \caption{Learning curves for optimized Deeper CNN showing training and validation metrics over epochs.}
    \label{fig:optimized_learning_curve}
\end{figure}

\subsubsection{Comparison with Previous Models}
Table II presents the performance comparison between our optimized Deeper CNN, the original Deeper CNN, and the ResNet-18 baseline.

\begin{table}[ht]
\caption{Performance Comparison of Original vs. Optimized Models}
\centering
\begin{tabular}{lccc}
\toprule
\textbf{Model} & \textbf{Test Accuracy} & \textbf{Training Time} & \textbf{Parameters} \\
\midrule
ResNet-18 (Transfer Learning) & 99.33\% & 32.48 min & 11.7 million \\
Optimized Deeper CNN & 92.33\% & 42.35 min & 9.4 million \\
Original Deeper CNN & 86.67\% & 36.06 min & 9.2 million \\
\bottomrule
\end{tabular}
\end{table}

The optimized Deeper CNN significantly narrowed the performance gap with the ResNet-18 model while maintaining the benefits of a custom-designed architecture. The architectural enhancements (particularly residual connections) and improved training strategy contributed most significantly to the performance gain, with only a marginal increase in parameter count.

\subsubsection{Impact of Individual Optimizations}
Through ablation studies, we identified the relative contribution of different optimization techniques:

\begin{itemize}
    \item \textbf{Residual Connections:} Provided the most substantial improvement (approximately 3.2\% accuracy gain) by enhancing gradient flow throughout the deep network.
    
    \item \textbf{Optimized Dropout Strategy:} The more gradual dropout progression improved accuracy by approximately 1.8\% compared to the original pattern.
    
    \item \textbf{Training Strategy:} The combination of AdamW optimizer and OneCycle learning rate scheduler contributed approximately 1.4\% to the accuracy improvement.
    
    \item \textbf{Data Augmentation:} The balanced approach to augmentation provided approximately 0.8\% accuracy gain.
\end{itemize}

These findings highlight the importance of architectural enhancements and training strategy optimization in improving model performance, even without resorting to transfer learning.

\subsection{Feature Visualization and Interpretation}
To better understand how our models learn to distinguish between instrument classes, we employed various visualization techniques:
\begin{itemize}
    \item Gradient-weighted Class Activation Mapping (Grad-CAM) to visualize regions of interest for classification decisions
    \item t-SNE visualization of feature embeddings to examine class separation
    \item Confusion matrix analysis to identify challenging instrument distinctions
\end{itemize}

\begin{figure}[ht]
    \centering
    % TODO: Insert visualization figure
    \caption{Visualization of model attention using Grad-CAM for various instrument classes, showing the regions that influence the classification decision.}
    \label{fig:gradcam}
\end{figure}

\subsubsection{Attention Mechanisms}
Grad-CAM visualizations revealed that:
\begin{itemize}
    \item \textbf{ResNet-18} demonstrates refined attention to specific discriminative parts of instruments like sound holes, keys, and string patterns.
    
    \item \textbf{Original Deeper CNN} tends to focus on broader regions of instruments, sometimes attending to contextual background elements.
    
    \item \textbf{Optimized Deeper CNN} shows more focused attention similar to ResNet-18, particularly for string instruments and wind instruments, suggesting that the architectural improvements enhanced feature localization.
\end{itemize}

\subsubsection{Feature Space Organization}
The t-SNE visualization of feature embeddings before the classification layer showed:
\begin{itemize}
    \item \textbf{ResNet-18} creates highly separated clusters with minimal overlap between classes.
    
    \item \textbf{Original Deeper CNN} produces less distinct boundaries between similar instrument families.
    
    \item \textbf{Optimized Deeper CNN} demonstrates improved class separation compared to the original model, particularly for brass instruments and string instruments, which were previously more frequently confused.
\end{itemize}

\subsubsection{Confusion Analysis}
Analysis of the confusion matrices revealed specific classification patterns:
\begin{itemize}
    \item Across all models, the most challenging distinctions were between related instruments within the same family (e.g., trumpet/trombone, violin/viola).
    
    \item The optimized Deeper CNN showed particular improvement in distinguishing between wind instruments and percussion instruments compared to the original model.
    
    \item Certain instrument pairs remained challenging even after optimization, suggesting that these might require additional specialized feature extraction.
\end{itemize}

These visualizations provided valuable insights into how our models learn to recognize musical instruments and helped identify specific areas where the optimizations improved the model's discrimination capabilities.
