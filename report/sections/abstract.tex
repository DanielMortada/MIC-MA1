% =====================================
% Abstract and Keywords
% File: sections/abstract.tex
% =====================================

\begin{abstract}
This paper presents a systematic three-phase investigation into the application of deep learning for musical instrument classification. Convolutional neural network (CNN) models were developed and evaluated for their ability to identify 30 distinct instrument classes from images. Initially, Phase 1 compared a ResNet-18 transfer learning baseline, which achieved 99.33\% test accuracy, against a custom CNN built from scratch, yielding 80.67\% accuracy. Subsequently, Phase 2 employed a flexible framework to systematically compare various custom CNN architectures—including deeper, wider, and regularized variants—ultimately identifying the Deeper CNN as the most effective custom configuration with 86.67\% accuracy. Finally, Phase 3 proceeded through an instructive two-stage optimization process: an initial approach that unexpectedly resulted in performance regression (81.00\%), followed by a refined strategy informed by failure analysis that successfully incorporated class-specific data augmentation, selective attention mechanisms, balanced regularization, and mixed precision training. Consequently, the optimized model attained 93.33\% test accuracy, significantly narrowing the performance gap relative to the transfer learning baseline. Key contributions encompass a thorough evaluation of the tradeoffs between transfer learning and bespoke architectures, empirical insights into optimization pitfalls and remedies, and the demonstration of how iterative refinement based on systematic failure analysis can substantially elevate the performance of custom models within specialized image classification domains.
\end{abstract}

\begin{keywords}
Deep Learning, Convolutional Neural Networks, Musical Instrument Classification, Computer Vision, Transfer Learning, ResNet, Architecture Optimization, Residual Connections.
\end{keywords}
