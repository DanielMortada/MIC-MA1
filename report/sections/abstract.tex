% =====================================
% Abstract and Keywords
% File: sections/abstract.tex
% =====================================

\begin{abstract}
This paper presents a systematic three-phase approach to musical instrument classification using deep learning techniques. We develop and evaluate convolutional neural network (CNN) models capable of identifying 30 distinct musical instrument classes from images. In Phase 1, we compare a transfer learning approach using ResNet-18 (achieving 99.33\% test accuracy) with a custom CNN architecture developed from scratch (achieving 80.67\% accuracy). In Phase 2, we employ a flexible experimental framework to systematically compare multiple CNN architectures, including deeper, wider, and more regularized variants, identifying the Deeper CNN as the best-performing custom architecture (86.67\% accuracy). Phase 3 focuses on optimizing this architecture through class-specific data augmentation, selective attention mechanisms, residual connections, graduated dropout strategies, and mixed precision training. Our optimized Deeper CNN achieves 93.33\% test accuracy, significantly closing the gap with the transfer learning approach. Our contributions include a comprehensive evaluation of the tradeoffs between transfer learning and custom architectures, empirical evidence on the relative importance of architectural components, and insights into effective optimization strategies. The results demonstrate that methodical architecture refinement and training optimization can substantially improve custom model performance for specialized image classification tasks.
\end{abstract}

\begin{keywords}
Deep Learning, Convolutional Neural Networks, Musical Instrument Classification, Computer Vision, Transfer Learning, ResNet, Architecture Optimization, Residual Connections.
\end{keywords}
